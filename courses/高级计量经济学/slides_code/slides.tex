\documentclass{beamer}

\usetheme{SUSTechbuildings}

\usepackage{amsmath}
\usepackage{natbib}
\usepackage{graphicx}
\usepackage{booktabs}
\usepackage{calligra}

\title{Finance and Corporate Innovation: A Survey}
\subtitle{--- FIN5016 Paper Presentation}
\institute{Group 10}
\author[FIN5016]{%
  \texorpdfstring{%
    \begin{columns}
      \column{.33\linewidth}
      \centering
      Chen Jiangrui\\
      12231319
      \column{.33\linewidth}
      \centering
      Li Dacheng\\
      12232959
      \column{.33\linewidth}
      \centering
      Li Huang\\
      12132978
    \end{columns}
  }{Chen Jiangrui \and Li Dacheng \and Li Huang}
}

\begin{document}

    \begin{frame}
        \titlepage
    \end{frame}

    \begin{frame}
        \tableofcontents
    \end{frame}

    \section{Introduction}

    \begin{frame}{Introduction}
        Corporate innovation is an increasingly important topic that has attracted great attention from academic researchers in financial economics in recent years.

        ~

        Related publications in the top three finance journals (\textit{JF}, \textit{JFE}, \textit{RFS}):
        \begin{itemize}
            \item 2000 -- 2008: 5
            \item 2009 -- 2017: \color{sustechorange}{56}
        \end{itemize}

        ~

        Largely because of the availability of high-quality patent and citation data that capture a country's or a firm's innovation output.
    \end{frame}

    \begin{frame}{Research Questions}
        \begin{enumerate}
            \item How is corporate finance motivated and financed?
            \item To what extent do financial markets and systems shape the initiation, process, features, and outcomes of technological innovation by corporations?
        \end{enumerate}

        ~

        Important to
        \begin{itemize}
            \item investors
            \item business practitioners
            \item social scientists
            \item policy makers
        \end{itemize}
    \end{frame}

    \begin{frame}{Importance of Innovation}
        Technological innovation is vital for a country's economic growth and a firm's long-term competitive advantage.

        ~

        \begin{itemize}
            \item Innovation accounts for approximately 50\% of a country's total GDP growth.
            \item Economists have estimated that 85\% of a nation's economic growth is attributable to technological innovation \citep{R2006InnovationGrowthTourism}.
            \item A one-standard deviation increase in patent stock per capital is associated with a 0.85\% increase in GDP growth \citep{CMZZ2018PatentsPortendProductivity}.
        \end{itemize}
        
    \end{frame}

    \section[Firm-Level Char.]{Firm-Level Characteristics}

    \subsection{Venture Capital and Entrepreneurship}

    \begin{frame}{Venture Capital and Entrepreneurship}

    \end{frame}

    \subsection{Firms' Internal Characteristics}

    \begin{frame}{Firms' Internal Characteristics}

    \end{frame}

    \subsection{Firms' External Characteristics}

    \begin{frame}{Firms' External Characteristics}

    \end{frame}

    \section[Market Char.]{Market Characteristics}

    \begin{frame}{Market Characteristics}
        
    \end{frame}
    
    \section[Institutional Features]{Institutional Features of a Society/Country}

    \subsection{Laws and Policies}

    \begin{frame}{Laws and Policies}
        \vspace{-1cm}
        \begin{itemize}
            \item Laws:
            \begin{itemize}
                \item IP (Intellectual Property) protection laws
                \item Labor laws
                \item Bankruptcy laws
            \end{itemize}
            \item Policy uncertainty
            \item ...
        \end{itemize}
    \end{frame}

    \begin{frame}{Laws and Policies}{IP Protection Laws}
        \vspace{-1cm}
        \begin{itemize}
            \item \citet{L2009EmpiricalImpactIntellectual} examines how IP protection laws affect innovation by analyzing 177 major patent policy shifts in 60 nations over the past 150 years.
            \item Policy changes include: 
            \begin{enumerate}
                \item Whether the country offered comprehensive patent protection;
                \item the length of time that a patent is valid;
                \item the cost associated with obtaining and maintaining the patent;
                \item provisions for patent revocation.
            \end{enumerate}
            \item Innovation measures: 
            \begin{enumerate}
                \item Patent filings in Great Britain by residents of the country;
                \item patent applications by domestic entities in the country;
                \item applications by foreign entities in that country.
            \end{enumerate}
        \end{itemize}
    \end{frame}

    \begin{frame}{Laws and Policies}{IP Protection Laws}
        \vspace{-1cm}
        \begin{itemize}
            \item Puzzle: Negative impact of IP protection law changes on the number of patents generated.
            \item Explanations:
            \begin{enumerate}
                \item The patent-based measure of innovation may not fully capture the true extent of innovation output.
                \item There might be confounding policy changes in some of the sample countries.
                \item The common wisdom among economists that patent protection can encourage innovative actions might be over-exaggerated.
            \end{enumerate}
        \end{itemize}
    \end{frame}

    \begin{frame}{Laws and Policies}{IP Protection Laws}
        \begin{itemize}
            \item \citet{FLW2017IntellectualPropertyRights} examine how IP rights protection affects innovation in China around the privatization of state-owned enterprises (SOEs).
            \item IP protection quality measure: A survey-based index published by the Chinese Academy of Social Sciences (CASS). The annual survey is to ask the respondents (legal professionals, e.g., judges and IPR lawyers, and corporate executives) to rate from 5 (best) to 1 (worse) three areas:
            \begin{enumerate}
                \item The length of time it takes for courts to resolve IP disputes;
                \item the cost of resolving the dispute as a percentage of the value of the IP under dispute;
                \item The fairness of court decisions.
            \end{enumerate}
            \item Innovation measure: weighted average patent number (new patents have large weights)
        \end{itemize}
    \end{frame}

    \begin{frame}{Laws and Policies}{IP Protection Laws}
        \vspace{-1cm}
        \begin{itemize}
            \item Finding: Innovation increases after SOE privatizations, and this increase is more pronounced in cities with strong IP rights protection.
            \item Implication: IP rights protection is beneficial to firms' innovative incentives but this positive effect mainly exists among non-SOE firms rather than SOEs.
        \end{itemize}
    \end{frame}

    \begin{frame}{Laws and Policies}{Labor Laws}
        \vspace{-1cm}
        \begin{itemize}
            \item Wrongful discharge laws: Protect employees against unfair firing/layoffs, limit firms' ability to hold up innovating employees after the innovation turns out to be successful.
            \item Intuition: By mitigating the possibility of hold-up risk faced by R\&D employees, such laws increase their incentives to innovate and in turn boost the employers' innovation output.
            \item \citet{ABS2014WrongfulDischargeLaws} formally test the above intuition and find that wrongful discharge laws indeed have a positive impact on innovation and new firm creation.
        \end{itemize}
    \end{frame}

    \begin{frame}{Laws and Policies}{Bankruptcy Laws}
        \vspace{-1cm}
        \begin{itemize}
            \item Bankruptcy laws: Protect the interests of creditors, on firms' incentives and efficacy in the innovation process.
            \item Intuition: When the bankruptcy code is friendly to creditors, innovative firms might be discouraged from pursuing innovation for fear of excessive liquidations. In contrast, a debtor-friendly bankruptcy code may lead to more innovation by promoting continuation upon failure.
            \item \citet{AS2009BankruptcyCodesInnovation} test and confirm the intuition above. They also find that the negative effect of a creditor-friendly bankruptcy code on innovation is more pronounced for firms in technologically innovative industries.
        \end{itemize}
    \end{frame}

    \begin{frame}{Laws and Policies}{Universal Demand Laws}
        \vspace{-1cm}
        \begin{itemize}
            \item Universal Demand Laws: Makes it harder for shareholders to file derivative lawsuits and thus reduces a company's shareholder litigation risk.
            \item \citet{LLM2021ShareholderLitigationCorporate} studied the impact of adoption of UD laws in 23 US states between 1989 and 2005 on firms' innovation activities and outcomes.
            \item Finding: Firms experience an increase in their innovation activities and outcomes after the adoption of UD laws.
        \end{itemize}
    \end{frame}

    \begin{frame}{Laws and Policies}{Policy Uncertainty}
        \vspace{-1cm}
        \begin{itemize}
            \item \citet{BHTX2017WhatAffectsInnovation} explore whether the uncertainty of government policies also affects corporate innovation.
            \item Policy uncertainty measure: National elections.
            \item Innovation measure: Patenting-based variables.
            \item Finding: Patenting outcomes significantly decrease during times of policy uncertainty, especially for more innovation-intensive industries.
            \item Explanation: Patenting outcomes decrease because number of inventors decrease during national elections.
        \end{itemize}
    \end{frame}

    \subsection{Financial Market Development}

    \begin{frame}{Financial Market Development}
        \vspace{-1cm}
        \begin{itemize}
            \item Financial systems
            \item International trade rules
            \item Trade liberalization
            \item Financial accounting regulation
        \end{itemize}
    \end{frame}

    \begin{frame}{Financial Market Development}{Financial Systems}
        \vspace{-1cm}
        \begin{itemize}
            \item \citet{T2006InnovationInformationFinancial} compares the innovation outcomes of industries operating in countries with bank-centered financial systems with those in countries with market-based systems.   
            \item Innovation measure: Increases in output yield due to shifts in the best-practice technology.
            \item Information intensiveness measure: Intangible assets fraction.
            \item Finding: While market-centered systems have a positive effect on innovations in almost all industrial sectors, bank-centered countries contribute more to innovation in information-intensive sectors.
        \end{itemize}
    \end{frame}

    \begin{frame}{Financial Market Development}{International Trade Rules}
        \begin{itemize}
            \item \citet{BDV2016TradeInducedTechnical} study the impact of Chinese import competition on innovation and productivity in 12 European countries.
            \item Import measure: Weighted average of imports across all sectors in which a firm operates.
            \item Innovation measures: 
            \begin{enumerate}
                \item Patenting-based variables;
                \item total-factor productivity (TFP);
                \item information technology (IT) intensity, e.g., computers per worker.
            \end{enumerate}
            \item Finding: The trade pressure induced by more Chinese imports stimulates firms to upgrade their technology and reallocate employment towards more innovative firms. In contrast, import competition from developed economies seems to have no significant effect on innovation.
        \end{itemize}
    \end{frame}

    \begin{frame}{Financial Market Development}{Trade Liberization}
        \vspace{-1cm}
        \begin{itemize}
            \item \citet{CMU2022BetterFasterStronger} studied the impact of trade policy during the Great Liberalization of the 1990s on innovation using firm-level patent data from over 65 countries.
            \item Liberalization measure: 
            \begin{itemize}
              \item Tariff redunctions $\implies$ Endogeneity
              \item Variation in applied most favored nation tariff cuts across a firm’s export markets $\implies$ Maybe exogenous to innovation
            \end{itemize}
            \item Innovation measure: Patenting.
            \item Finding: Trade liberalization has a positive, causal effect on corporate innovation in terms of new knowledge generation.
        \end{itemize}
    \end{frame}

    \begin{frame}{Financial Market Development}{Financial Accounting Regulation}
        \vspace{-1cm}
        \begin{itemize}
            \item \citet{LTMZ2016RealEffectFinancial} explore how International Financial Reporting Standards (IFRS) affect corporate innovation.
            \item Innovation measures:
            \begin{enumerate}
                \item Number of granted patents $\implies$ Innovation quantity
                \item Number of citations received by patents $\implies$ Innovation quality
            \end{enumerate}
            \item Finding: Mandatory IFRS adopters experience a substantial increase in innovation output during the post-IFRS adoption period.
            \item Explanations: 
            \begin{enumerate}
                \item Relaxed financial constraints
                \item Improved managerial learning from stock prices induced by IFRS
            \end{enumerate}
        \end{itemize}
    \end{frame}

    \subsection{Demographic and Social Traits of a Country or a Region}

    \begin{frame}{Demographic and Social Traits of a Country or a Region}
        \vspace{-1cm}
        \begin{itemize}
            \item Religiosity
            \item Sexual orientation
            \item Gambling preferences
            \item Bribery
        \end{itemize}
    \end{frame}

    \begin{frame}{Demographic and Social Traits of a Country or a Region}{Religiosity}
        \vspace{-1cm}
        \begin{itemize}
            \item \citet{BTV2022ForbiddenFruitsPolitical} uncover a robust negative relationship between religiosity and patents per capita in international and U.S. data.
            \item Religiosity measures (survey-based, all from World Values Survey):
            \begin{enumerate}
                \item Religious person fraction;
                \item Belief in god fraction;
                \item Church attendance fraction.
            \end{enumerate}
            \item Innovation measure: Patents per capita
        \end{itemize}

    \end{frame}

    \begin{frame}{Demographic and Social Traits of a Country or a Region}{Sexual Orientation}
        \vspace{-1cm}
        \begin{itemize}
            \item \citet{GZ2017EmploymentNondiscriminationActs} studied the impact of US state-level Employment Non-Discrimination Acts (ENDAs) on corporate innovation.
            \item Innovation measures: 
            \begin{enumerate}
                \item Number of granted patents $\implies$ Innovation quantity
                \item Number of citations received by patents $\implies$ Innovation quality
            \end{enumerate}
            \item Finding: ENDAs encourage corporate innovation.
            \item Explanation: ENDAs match innovative firms with pro-gay employees who are typically more creative than anti-gay employees.
        \end{itemize}
    \end{frame}

    \begin{frame}{Demographic and Social Traits of a Country or a Region}{Gambling Preferences}
        \begin{itemize}
            \item \citet{CPRV2014LocalGamblingPreferences} find that firms headquartered in countries in which gambling propensity is higher tend to undertake riskier projects, spend more on innovation, and generate greater innovative output.
            \item Gambling preferences measure: Catholics-to-Protestants ratio (Catholic population relative to the Protestant population)
            \begin{itemize}
                \item Protestants are typically fervently opposed to all forms of gambling, while Catholics tend to be more tolerant of gambling practices.
            \end{itemize}
            \item Innovation measure:
            \begin{enumerate}
                \item Research input: R\&D spending scaled by book assets
                \item Research output: Patents and citations
            \end{enumerate}
            \item Explanation: Investment in innovation makes a company's stock price more lottery-like, which is a feature desired by individuals who love gambling.
        \end{itemize}
    \end{frame}

    \begin{frame}{Demographic and Social Traits of a Country or a Region}{Bribery}
        \vspace{-1cm}
        \begin{itemize}
            \item \citet{ADM2014BribePaymentsInnovation} studied the relationship between innovation and bribery in firms.
            \item Bribery measure: firm responses to the question “What percent of annual sales value does a typical firm like yours spend on gifts or informal payments to public officials to ‘get things done’ with regard to customs, taxes, licenses, regulations, services, etc.?”
            \item Innovation measure: A dummy variable which takes the value of 1 if the firm developed a new product line, and 0 otherwise.
            \item Finding: Innovating firms pay more bribes than non-innovating firms, especially in those countries with more bureaucratic regulation and weaker governance.
        \end{itemize}
    \end{frame}

    \section[Future Directions]{Future Directions or Studies on Finance and Corporate Innovation}

    \begin{frame}{Future Directions or Studies on Finance and Corporate Innovation}{New Innovation Measure}
        \vspace{-0.5cm}
        \begin{itemize}
            \item Future direction: New empirical proxies that better capture the extent of corporate innovation activities than self-reported R\&D expenditures and patenting-based measures.
            \item Limitations of R\&D expenditures:
            \begin{enumerate}
                \item R\&D expenditures \textbf{only capture one particular observable quantitative input} and cannot capture the different dimensions of a firm's innovation strategies.
                \item R\&D is \textbf{sensitive to accounting norms} such as whether they should be capitalized or expensed.
                \item Information on self-reported R\&D expenditures contained in financial statements (e.g. those from the Compustat database) is \textbf{unreliable}, which may introduce a significant measurement error problem.
            \end{enumerate}
        \end{itemize}
    \end{frame}

    \begin{frame}{Future Directions or Studies on Finance and Corporate Innovation}{New Innovation Measure}
        \vspace{-0.5cm}
        \begin{itemize}
            \item Limitations of patenting-based measures:
            \begin{enumerate}
                \item Patent-based measures of innovation \textbf{may not fully capture the true extent of innovation output} after observing a few puzzling empirical findings with regard to patent-based measures.
                \item \textbf{Patenting is just one way to protect a firm's intellectual property}, which largely depends on its own discretion and strategic plans. For example, many corporate innovation outputs take the form of trade secrets because their developers do not want to file for patents.
                \item \textbf{Patent data itself have a few problematic features}, such as truncation issues, the difficulty of adjusting for technology classes, the vast disparity in innovative activities across regions, and misleading assignment practices, etc., which may lead to erroneous conclusions if these issues are not properly addressed.
            \end{enumerate}
        \end{itemize}
    \end{frame}

    \begin{frame}{Future Directions or Studies on Finance and Corporate Innovation}{Recent Attempts}
        \vspace{-1cm}
        \begin{itemize}
            \item \citet{KPSS2017TechnologicalInnovationResource} propose a new way of measuring the value of innovation outcomes, that is, the market-perceived value of patents at the time of granting.
            \item \citet{CKY2018MeasuringInnovation} propose a firm's R\&D quotient, defined as the firmspecific output elasticity of R\&D expenditures, as an alternative corporate innovation proxy.
            \item \citet{BBC2021TextBasedAnalysisCorporate} develop a new proxy for corporate innovation by conducting a textual analysis of financial analysts' reports.
        \end{itemize}
    \end{frame}

    \section{Conclusion}

    \begin{frame}{Conclusion}

    \end{frame}

    \begin{frame}[allowframebreaks]{References}
        \vspace{-0.5cm}
        \bibliographystyle{ecta}
        {\small
        \bibliography{mybib}
        }
    \end{frame}

    \begin{frame}
        \centering
        \Huge\calligra Thanks!
    \end{frame}
\end{document}